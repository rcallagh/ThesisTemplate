\message{ !name(../main.tex)}%\newcommand{\finalversion}{}
\input{template/template}

\graphicspath{ {src/} }


\includeonly{
   src/acronyms,
   src/plan,
   src/introduction,
   src/background,
   src/conclusions,
 }



\newcommand{\titleinfo}{Simulations of diffusion MR (get proper title) }

\pagenumbering{Alph}
\begin{document}

\message{ !name(src/introduction.tex) !offset(-22) }
\chapter{Introduction}
\label{sec:introduction}

This report summarises work towards improving the realism simulations of \ac{MRI}. Realistic simulations allow models of the MRI signal to be validated using controllable and well known ground truth. 

MRI provides researchers and clinicians a powerful and flexible tool for non-invasively imaging the human body \emph{in vivo} and has found extensive use over the past few decades in furthering the understanding the structure and function of the human brain.
One technique which is commonly employed to study the structure of the human brain is diffusion MRI (dMRI).

dMRI sensitises the MRI signal to the motion of water molecules which is known as diffusion.
The environment in which the water molecules move will restrict the motion of the molecules and so will affect the MRI signal. 
This dependency of the dMRI signal on the environment in which water molecules diffuse can be exploited to infer information about the environment solely from the dMRI signal.  
Microstructure imaging attempts to do exactly this, infer information about the microstructural environment such as cell size and density from the dMRI signal.
In order to infer meaningful information from the dMRI signal, models are typically used which relate microstructural features to the dMRI signal.

The validation of these microstructural models can be difficult since ground truth microstructural features are inaccessible \emph{in vivo} and classical histological techniques have limitations such as disruption due to tissue extraction and preparation. 
One approach commonly taken for the validation of new models is simulation of the diffusion MRI signal using well defined ground truth microstructural environments known as numerical phantoms. 

Whilst these numerical often provide a valuable ground truth for simulation, they typically over simplify the complex microstructure of real tissue.
An example of this is in white matter (WM), where fibres are commonly represented as straight cylinders whereas in real tissue, fibres have complex shapes with undulation and diameter variation.

The main aim of this work is to improve the realism of WM numerical phantoms to more closely match \emph{in vivo} tissue.

The rest of the report is arranged as follows: \Cref{sec:background} outlines some of the physics behind diffusion MRI and how we simulate it. \Cref{sec:literature_review} reviews current literature on diffusion MRI simulations and numerical phantom generation. \Cref{sec:method} outlines the method we have chosen for generating new phantoms and some experiments. \Cref{sec:results} shows examples of the realistic phantoms we generate and the results of the simulation experiments carried out and \Cref{sec:discussion,sec:conclusion} discuss the results, limitations and future perspectives for the work.


%%% Local Variables:
%%% mode: latex
%%% TeX-master: "../main"
%%% End:

\message{ !name(../main.tex) !offset(82) }

\end{document}

%%% Local Variables:
%%% mode: latex
%%% TeX-master: t
%%% End:
