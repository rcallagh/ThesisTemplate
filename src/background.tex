\renewcommand{\BrainFuckChapter}{y}
\renewcommand{\LifeChapter}{y}
\chapter{Background}
\label{chap:background}
\chaptertoc{}

\begin{chapterabstract}
This is the brief description of the background chapter
\end{chapterabstract}

\section{A section}
\label{sec:bg_mri_physics}
\blindtext

\subsection{A subsection}
\label{sec:bg_nuclearmagnetism}
\blindmathpaper

\subsection{A subsection with an algorithm}
\label{sec:bg_algorithm}
Here is an algorithm.
\begin{algorithm}
  \begin{algorithmic}
    \State Do something
    \State Do another thing
    \While{test is true}
    \State Do this thing
    \State Do another thing
    \EndWhile
    \State End the thing
  \end{algorithmic}
  \caption{Basic algorithm listing.}
  \label{alg:MC_random_walk}
\end{algorithm}

\subsection{A subsection showing acronyms and citations}
\label{sec:bg_acronyms}
In this section, I demonstrate how acronyms and citations work in my thesis. Acronyms are defined in \texttt{src/acronyms.tex}. For instance here is \ac{EA}, the first time an acronym is called, it's written in full but after that it is abbreviated like so, \ac{EA}.

Citations are made in the usual LaTeX manner, stored in \texttt{references.bib}. Here's an example \cite{ref1,ref2}.

You can cross reference sections, equations etc. using \texttt{cleveref}. e.g. \Cref{sec:bg_algorithm}.
%%% Local Variables:
%%% mode: latex
%%% TeX-master: "../main"
%%% End:
