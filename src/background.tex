\renewcommand{\BrainFuckChapter}{y}
\renewcommand{\LifeChapter}{y}
\chapter{Background}
\label{chap:background}
\chaptertoc{}

\begin{chapterabstract}
This is the brief description of the background chapter
\end{chapterabstract}

\section{A section}
\label{sec:bg_mri_physics}
\blindtext

\subsection{A subsection}
\label{sec:bg_nuclearmagnetism}
\blindmathpaper

\subsection{A subsection with an algorithm}
\label{sec:bg_algorithm}
Here is an algorithm.
\begin{algorithm}
  \begin{algorithmic}
    \State Do something
    \State Do another thing
    \While{test is true}
    \State Do this thing
    \State Do another thing
    \EndWhile
    \State End the thing
  \end{algorithmic}
  \caption{Basic algorithm listing.}
  \label{alg:MC_random_walk}
\end{algorithm}

\subsection{A subsection showing acronyms and citations}
\label{sec:bg_acronyms}
In this section, I demonstrate how acronyms and citations work in my thesis. Acronyms are defined in \texttt{src/acronyms.tex}. For instance here is \ac{EA}, the first time an acronym is called, it's written in full but after that it is abbreviated like so, \ac{EA}.

Citations are made in the usual LaTeX manner, stored in \texttt{references.bib}. Here's an example \cite{ref1,ref2}.

You can cross reference sections, equations etc. using \texttt{cleveref}. e.g. \Cref{sec:bg_algorithm}.

\subsection{A subsection describing the chapter header images}
The chapter header images are defined in the \texttt{figures/chapter\_headers.pdf}. Basically each page is an image for a different chapter, taken in order. If you want to change it, the update pdf pages should be the same size as they are now. You can turn it off by using \verb|\renewcommand{LifeChapter}{}| instead of \verb|\renewcommand{LifeChaper}{y}| at the top of the file.
%%% Local Variables:
%%% mode: latex
%%% TeX-master: "../main"
%%% End:
