\chapter{Conclusions}
\label{chap:conclusion}

\chaptertoc{}

\begin{chapterabstract}
The purpose of this chapter is to summarise the contributions made by the work presented in this thesis, including the new white matter numerical phantom generator, \ac{ConFiG}, and discuss future directions for the project.
\end{chapterabstract}


\section{Contributions}
\label{sec:conc_contributions}
In this thesis, we have presented a new method for generating realistic numerical phantoms of \acf{WM} for \acf{dMRI} simulations.
Our primary contribution is the development of \acf{ConFiG}, a novel approach to phantom generation in which we mimic the processes governing the growth of real axons in an effort to produce realistic \ac{WM} phantoms.
We demonstrated, though a preliminary implementation and subsequent improvements, that \ac{ConFiG} is able to produce phantoms with state-of-the-art performance in terms of the axonal density accomplished, by following simple rules inspired by the biological processes governing axonal guidance.

Diffusion MRI simulations show that the microstructure generated by \ac{ConFiG} can be used to generate realistic synthetic \ac{dMRI} data, acting as a proof-of-concept of the utility of \ac{ConFiG} phantoms.
Comparisons to real axons segmented from \ac{EM} show that \ac{ConFiG} produces axons with similar morphologies to real axons, capturing not just axon diameter and orientation distributions, but also subtle features such as variable diameters along axons and non-circular cross-sections. Taking all of this into account, \ac{ConFiG} phantoms represent a new generation of \ac{WM} numerical phantoms, greatly increasing phantom realism when compared to the current standard of \ac{WM} numerical phantoms which are based on cylinders.

As an example application of \ac{ConFiG}, we probed assumptions that are inherent in \acf{SD} based \ac{dMRI} modelling techniques.
We were able to demonstrate that microscopic variations in the shapes of individual axons within a bundle mean that the intra-axonal signal varies per-fibre, violating one of the assumptions in \ac{SD} techniques that each fibre has an identical \ac{dMRI} response. The variability in the \ac{FRF} also leads to a variability in the \ac{FOD} estimated which could have a large impact \ac{FOD} based techniques such as tractography.
We further demonstrated that some signal from the extracellular space remains, even at high $b$-values ($b>\SI{3000}{\second\per\milli\metre\squared}$), which is contrary to assumptions made in some \ac{SD} based techniques such as \acf{AFD}.
The extracellular signal also affects \ac{FOD} estimation since the shape of the extracellular signal can vary in different environments, leading to differences in the \ac{FOD} when the same \ac{FRF} is used.



\section{Future directions}
\label{sec:conc_future_directions}
\ac{ConFiG} opens up many possibilities for future investigation by giving us the ability to generate more realistic \ac{WM} phantoms than ever before. One such possibility is to study the dynamics of the diffusion process itself. For instance, we could study the diffusion time dependence in realistic \ac{WM} phantoms, similarly to recent work investigating intracellular diffusion \cite{Andersson2020,Lee2019a}, however \ac{ConFiG} offers us the potential to study the realistic extracellular space too. This could help to disentangle different sources of diffusion time dependence (axonal beading, undulation, extracellular space etc.) and identify which source is the most significant.


Further, we can investigate how best to sensitise the \ac{dMRI} signal to these complex microstructural features. Recent advances in \ac{dMRI} acquisition have given us novel diffusion encoding schemes such as b-tensor encoding \cite{Westin2016} which provide additional information to standard \ac{PGSE} techniques.
While this gives us a new tool to study microstructure, the potential space of possible imaging schemes becomes very large, making it difficult to know which schemes may be best to identify which features. 
This is something that \ac{ConFiG} gives us an opportunity to empirically test, by generating a wide range of phantoms covering different microstructural environments and simulating various \ac{dMRI} sequences to test which sequences or combination of sequences give us the best sensitivity.

\ac{ConFiG} also offers us a new tool for validating existing models of \ac{dMRI} microstructure. Previous models for measuring microstructural features such as axon diameter or orientation dispersion have used simple cylinder models in simulation validations \cite{Alexander2017,Zhang2012,Zhang2011,Alexander2010}, \ac{ConFiG} enables us to test these models in realistic geometries.
In principle, this offers us the possibility for a more comprehensive validation of \ac{dMRI} modelling techniques, it will require some thought because certain microstructural features are not intuitively defined in these complex phantoms.
For instance, the fibre diameter is simple to define when using cylinders, but when you have complex fibres that not only have a variable diameter along the fibres but also non-circular cross-sections, such a thing becomes more difficult to define.
\ac{ConFiG} phantoms may enable us to probe some of these questions, however, to find out which features can be best estimated using \ac{dMRI}, for instance are we more sensitive to diameter variation along fibres (i.e beading) or the overall mean diameter.

The experiments presented in \Cref{chap:frf_experiment} demonstrate an example of this kind of application of \ac{ConFiG} to investigate a \ac{dMRI} mode, demonstrating that certain assumptions in \ac{SD} techniques may not hold true.
Beyond this application to simply probe \ac{dMRI} models, we may use this to develop improved models. For instance, it may be possible to include variability in the \ac{FRF} into the modelling process to try to capture the variations in axonal morphology.
Beyond \ac{SD} techniques, there are many \ac{dMRI} modelling techniques which take a similar approach to the representation of the \ac{dMRI} signal, splitting it into a diffusion response term and an orientation distribution term.
An example of this is \acf{NODDI} \cite{Zhang2012}, which uses a Watson distribution for the orientation dispersion distribution and the MR response is diffusion in sticks, with an extracellular and CSF compartment. 
As in the \ac{SD} case however, \ac{NODDI} uses the same basic MR response for all the fibres (that is, they're all sticks with the same diffusivity), while we have shown that fibres have different responses. Again, it may be possible to attempt to account for this variability when modelling \ac{dMRI} using models such as \ac{NODDI}.

Another potential application of \ac{ConFiG} is in the development of a new computational model of \ac{WM} microstructure.
The idea here would be to estimate microstructural features by learning a mapping between \ac{dMRI} signals and microstructure.
This could be done by generating a wide range of \ac{WM} numerical phantoms spanning the range of possible expected microstructural environments in the brain and simulating the \ac{dMRI} signal within them to generate a dictionary of signals paired with ground-truth microstructure.
As mentioned above, \ac{ConFiG} phantoms may be used to identify the set of \ac{dMRI} sequences that give us the best sensitivity to various microstructure features to give this computational model the best chance to work effectively.
The benefit of developing this computational model would be that it can capture the relationship between complex microstructure and \ac{dMRI} signal without relying on an explicit analytical model and further may be able to identify features which do not have a simple analytical expression such as axonal beading.
Such approaches have recently been used to estimate membrane permeability, one such feature that is hard to model analytically \cite{Hill2019,Nedjati-Gilani2017}.

\ac{ConFiG} itself can be also be improved. Currently the maximum size of a phantom is limited to around \SI{50 x 50 x 50}{\micro\metre} which limits the diffusion time that can be explored without needing to extend phantoms with replicated copies. This is down to memory and computational time constraints from using a large number of points in the growth network.
As mentioned in \Cref{sec:config_discussion}, this could be addressed by growing in layers enabling dense sampling of small regions of space using fewer points.
Another option is to reconfigure the algorithm to remove the reliance on the network.
Currently the network plays two main roles, one is defining the paths available for a fibre to grow and one is sampling the space so that growing fibres know where existing fibres are.
A potentially better approach is to separate these two processes so that the sampling of the space is done by one structure (potentially a Deluanay/Voronoi based approach as currently or potentially something like an Octree or \ac{BVH}\cite{ericsonRealTimeCollisionDetection2004}) and the paths are left separate to enable fibres to have more freedom in growth directions.
This approach could potentially enable faster and more memory-efficient growth of \ac{ConFiG} phantoms while giving the fibres more freedom in paths could enable even high density of phantoms.

The potential future directions presented here are only limited to applications of \ac{ConFiG} to \ac{dMRI} of \ac{WM}. Since the growth idea behind \ac{ConFiG} is quite general, it is easy to see how \ac{ConFiG} could be adapted to apply to other tissues (for instance addin branching cells and vasculature) by changing the growth rules.
Additionally, since the end product of \ac{ConFiG} is a 3D surface mesh, these could be used in any other simulator that can handle a 3D mesh. Very basic examples of this are presented throughout the thesis as the meshes are used in Blender to render (effectively simulating visible light imaging) 3D images and 2D `virtual histology' images.
All of this means that \ac{ConFiG} represents a potentially powerful new tool for the \ac{dMRI} community and beyond.


%%% Local Variables:
%%% mode: latex
%%% TeX-master: "../main"
%%% End:
