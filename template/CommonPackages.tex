%%% File encoding is ISO-8859-1 (also known as Latin-1)

% Input encoding is 'latin1' (Latin 1 - also known as ISO-8859-1)
% CTAN: http://www.ctan.org/pkg/inputenc
%
% A newer package is available - you may look into:
% \usepackage[x-iso-8859-1]{inputenc}
% CTAN: http://www.ctan.org/pkg/inputenx
\usepackage[utf8]{inputenc}

% CTAN: http://www.ctan.org/pkg/fontenc
\usepackage[T1]{fontenc}

% Language support for 'english' (alternative 'ngerman' or 'french' for example)
% CTAN: http://www.ctan.org/pkg/babel
\usepackage[english]{babel}

% Doing calculations with LaTeX units -- needed for the vertical line in the footer
% CTAN: http://www.ctan.org/pkg/calc
\usepackage{calc}
\usepackage[nomessages]{fp}% http://ctan.org/pkg/fp

% Extended graphics support
% There is also a package named 'graphics' - watch out!
% CTAN: http://www.ctan.org/pkg/graphicx
\usepackage{graphicx}

% Extendes support for floating objects (tables, figures), adds the [H] placing option (\begin{figure}[H]) which palces it "Here" (without any doubt).
% CTAN: http://www.ctan.org/pkg/float
\usepackage{float}

% Extended color support
% I use the command \definecolor for example.
% Option 'Table': Load the colortbl package, in order to use the tools for coloring rows, columns, and cells within tables.
% CTAN: http://www.ctan.org/pkg/xcolor
\usepackage[table,svgnames]{xcolor}

% Nice tables
% CTAN: http://www.ctan.org/pkg/booktabs
\usepackage{booktabs}

% Better support for ragged left and right. Provides the commands \RaggedRight and \RaggedLeft.
% Standard LaTeX commands are \raggedright and \raggedleft
% http://www.ctan.org/pkg/ragged2e
\usepackage{ragged2e}

%\usepackage{cleveref}

% Create function plots directly in LaTeX
% CTAN: http://www.ctan.org/pkg/pgfplots

%%%%%%%% Packages %%%%%%%%%%%
\usepackage[printonlyused]{acronym}
%\renewcommand*{\acsfont}[1]{\textit{#1}}
%\renewcommand*{\acffont}[1]{\textbf{{#1}}}

%\usepackage{acro}

\usepackage{needspace}

\usepackage{amsmath}
\usepackage{xfrac}
\usepackage{mhchem}
\usepackage{gensymb}%For \degree

\usepackage{blindtext}
\usepackage{comment}
\usepackage{graphicx}
\usepackage{helvet}
\usepackage{times}

\usepackage{listings}
\lstset{
  language=python,
  literate=%
    {0}{{{\color{lime!50!black}0}}}1
    {1}{{{\color{lime!50!black}1}}}1
    {2}{{{\color{lime!50!black}2}}}1
    {3}{{{\color{lime!50!black}3}}}1
    {4}{{{\color{lime!50!black}4}}}1
    {5}{{{\color{lime!50!black}5}}}1
    {6}{{{\color{lime!50!black}6}}}1
    {7}{{{\color{lime!50!black}7}}}1
    {8}{{{\color{lime!50!black}8}}}1
    {9}{{{\color{lime!50!black}9}}}1,
  basicstyle=\sffamily,
  keywordstyle=\sffamily\bfseries\color{orange!60!black},
  identifierstyle=\color{teal!40!black},
}

\usepackage{makeidx}
\usepackage{multirow}

\usepackage{url}


\usepackage{float}
\usepackage{caption}
\usepackage{subcaption}

\usepackage{etoolbox}
\usepackage{pgfgantt}
\usepackage{rotating}
\usepackage{relsize}

\providecommand{\description}{}
\usepackage{paralist}
\usepackage{enumerate}
\usepackage{enumitem}

\makeatletter
\@ifundefined{previewmode}{}{
\usepackage[tightpage,active,noconfig]{preview}
}
\makeatother

\usepackage[normalem]{ulem}

\ifx\finalversion\undefined
\usepackage[color=green!40, colorinlistoftodos, figwidth=\columnwidth]{todonotes}
\newcommand{\todoI}[1]{

\resizebox{0.95\columnwidth}{!}{\todo[inline]{#1}}}
\else
\newcommand{\todoI}[1]{}
\newcommand{\todo}[1]{}
\newcommand{\listoftodos}{}
\fi


